\documentclass[11pt]{article}
\usepackage{datetime}
\usepackage{color,array,graphics}
\usepackage{enumerate}
\usepackage[pdftex, colorlinks, linkcolor=red,citecolor=red,urlcolor=blue]{hyperref}
\usepackage{ulem}
\usepackage{xcolor}  % For setting text color
\usepackage{marginnote}  % For additional margin note features



\setlength{\parindent}{0cm}

\setlength{\parskip}{0.3cm plus4mm minus3mm}

\textwidth  6.5in
\oddsidemargin +0.0in
\evensidemargin +0.0in
\textheight 9.0in
\topmargin -0.5in

\usepackage{upquote,textcomp}
\usepackage{amssymb,amsmath,amsfonts,amsthm}
\usepackage{graphicx}
\usepackage{multicol}

\newcommand{\vect}[1]{{\bf #1}}                 %for bold chars
\newcommand{\vecg}[1]{\mbox{\boldmath $ #1 $}}  %for bold greek chars
\newcommand{\matx}[1]{{\bf #1}}

\def\OR{\vee}
\def\AND{\wedge}
\def\imp{\rightarrow}
\def\math#1{$#1$}

\DeclareSymbolFont{AMSb}{U}{msb}{m}{n}
\DeclareMathSymbol{\N}{\mathbin}{AMSb}{"4E}
\DeclareMathSymbol{\Z}{\mathbin}{AMSb}{"5A}
\DeclareMathSymbol{\R}{\mathbin}{AMSb}{"52}
\DeclareMathSymbol{\Q}{\mathbin}{AMSb}{"51}
\DeclareMathSymbol{\I}{\mathbin}{AMSb}{"49}
\DeclareMathSymbol{\C}{\mathbin}{AMSb}{"43}

\begin{document}
\thispagestyle{empty}

\begin{center}
\large
\textbf{CSCI 2200 --- Foundations of Computer Science (FoCS) \\
Homework 2 (document version 1.0)}
\end{center}

\section*{Overview}
\begin{itemize}
\item This homework is due by 11:59PM on Thursday, September~28
\item You may work on this homework in a group of up to four students;
  unlike recitation problem sets,
  \textbf{your teammates may be in any section}
\item You may use at most \textbf{two} late days on this assignment
\item Please start this homework early and ask questions during
  office hours; % and at your September~13 recitation section;
  also ask questions on the Discussion Forum
\item Please be concise in your written answers;
  even if your solution is correct, if it is not well-presented,
  you may still lose points
\item You can type or hand-write (or both) your solutions
  to the required graded problems below;
  \textbf{all work must be organized in one PDF that lists
  all teammate names}
\item You are strongly encouraged to use LaTeX, in particular for
  mathematical symbols;
  see the corresponding \verb+hw1.tex+ file as a starting point
  and example
\end{itemize}

%\vspace{0.2in}

\section*{Grading}
\begin{itemize}
\item For each assigned problem, a grade of 0, 1, or 2 is assigned
  as follows:
  0 indicates no credit;
  1 indicates half credit;
  and 2 indicates full credit
\item No credit is assigned if a problem is not attempted
  or minimal work/progress is shown
\item Half credit is assigned if a strong attempt was made
  toward a solution and/or only part of the problem was attempted or solved
\item Full credit is assigned for a perfect or nearly perfect solution,
  i.e.,~only one or two minor typos/mistakes at most
\end{itemize}

\vspace{0.2in}
\newpage
\section*{Warm-up exercises}
The problems below are good practice problems to work on.
Do not submit these as part of your homework submission.
\textbf{These are ungraded problems.}

\begin{multicols}{2}
\begin{itemize}

\item \textbf{Problem 3.32.}
\item \textbf{Problem 3.49.}
\item \textbf{Problem 4.11.}
\item \textbf{Problem 4.14.}
\item \textbf{Problem 4.16.}
\item \textbf{Problem 4.47.}
\item \textbf{Problem 4.48(c).} (See Problem 4.47.)
\item \textbf{Problem 5.10.}
\item \textbf{Problem 6.3(a).}
\item \textbf{Problem 6.15.}
\item \textbf{Problem 6.32.}
\item \textbf{Problem 7.3.}
\item \textbf{Problem 7.4(a-c).} \\
(In each subproblem, remove the recursion in your formula for $A_n$.)

\end{itemize}
\end{multicols}

\section*{Graded problems}
The problems below are required and will be graded.
\begin{itemize}

\item \textbf{*Problem 3.59 (Closure).}
\item \textbf{*Problem 4.10(k-l).}
\item \textbf{*Problem 5.12(d).}
\item \textbf{*Problem 5.20.}
\item \textbf{*Problem 5.39.}
\item \textbf{*Problem 6.8.}
\item \textbf{*Problem 6.43.}

\end{itemize}

As you might not have the required textbook yet,
all of the above problems (both graded and ungraded)
are transcribed in the pages that follow.

Graded problems are noted with an asterisk~(*).

If any typos exist below, please use the textbook description.

\newpage
\begin{itemize}

\item \textbf{Problem 3.32.}
  Use truth tables to verify the rules for derivations in Figure~3.1 on page~29.
  Now use the rules in Figure~3.1 to show logical equivalence
  $$\neg((p\AND q)\OR r)\overset{\text{eqv}}{\ \equiv\ }(\neg p\AND\neg r)\OR(\neg q\AND \neg r)\text{.}$$

\vspace{0.1in}

\item \textbf{Problem 3.49.}
  What is the difference between
  $$\forall x: (\neg\exists y: P(x)\imp Q(y))\ \ \text{and}\ \ \neg\exists y: (\forall x: P(x)\imp Q(y))\text{?}$$

\vspace{0.1in}

\item \textbf{*Problem 3.59 (Closure).}
  A set $\mathcal{S}$ is closed under an operation if performing that
  operation on elements of $\mathcal{S}$ returns an element in $\mathcal{S}$.
  Here are five examples of closure.

  \begin{center}
  $\mathcal{S}$ is closed under addition $\imp$
  $\forall(x,y)\in\mathcal{S}^2: x+y\in\mathcal{S}$.

  $\mathcal{S}$ is closed under subtraction $\imp$
  $\forall(x,y)\in\mathcal{S}^2: x-y\in\mathcal{S}$.

  $\mathcal{S}$ is closed under multiplication $\imp$
  $\forall(x,y)\in\mathcal{S}^2: xy\in\mathcal{S}$.

  $\mathcal{S}$ is closed under division $\imp$
  $\forall(x,y\ne 0)\in\mathcal{S}^2: x/y\in\mathcal{S}$.

  $\mathcal{S}$ is closed under exponentiation $\imp$
  $\forall(x,y)\in\mathcal{S}^2: x^y\in\mathcal{S}$.
  \end{center}

  Which of the five operations are the following sets closed under?
  (a)~$\N$.
  (b)~$\Z$.
  (c)~$\Q$.
  (d)~$\R$.

  \subsubsection*{(a)~Natural Numbers}
  \begin{itemize}
    \item Closed Under Addition - Sum of two natural numbers will always be another natural number
    \item Not Closed Under Subtraction - Difference of two natural numbers will not always be another natural number (i.e., 9 - 12 = -3)
    \item Closed Under Multiplication - Product of two natural numbers will always be another natural number
    \item Not Closed Under Division - Quotient of two natural numbers will not always be another natural number (i.e., 9/2 = 4.5 )
    \item Closed Under Exponentiation - A natural number to the power of another natural number will always result in a positive integer, making it natural.
  \end{itemize}
  \subsubsection*{(b)~Integers}
  \begin{itemize}
    \item Closed Under Addition - Sum of two integers will always be another integer
    \item Closed Under Subtraction - Difference of two natural numbers will always be another integer
    \item Closed Under Multiplication - Product of two integers will always be another integer
    \item Not Closed Under Division - Quotient of two integers will not always be another integer (i.e., 9/2 = 4.5 )
    \item Not Closed Under Exponentiation - A integer to the power of another integer will not always result in another integer (i.e., \(2^{-3}\) = 0.125)
  \end{itemize}
  \subsubsection*{(c)~Rational Numbers}
  \begin{itemize}
    \item Closed Under Addition - Sum of two rational numbers will always be another rational number
    \item Closed Under Subtraction - Difference of two rational numbers will always be another rational number
    \item Closed Under Multiplication - Product of two rational numbers will always be another rational number 
    \item Closed Under Division - Quotient of two rational numbers will always be another rational number
    \item Not Closed Under Exponentiation - A rational number to the power of another rational number will not always result in another rational number (i.e., \(\left(2\right)^{\frac{1}{2}} = \sqrt{2}\))
  \end{itemize}
  \subsubsection*{(d)~Real Numbers}
  \begin{itemize}
    \item Closed Under Addition - Sum of two real numbers will always be another real number
    \item Closed Under Subtraction - Difference of two real numbers will always be another real number
    \item Closed Under Multiplication - Product of two real numbers will always be another real number 
    \item Closed Under Division - Quotient of two real numbers will always be another real number
    \item Closed Under Exponentiation - A real number to the power of another real number will always result in another real number
  \end{itemize}

\vspace{0.1in}

\item \textbf{*Problem 4.10(k-l).}
  You may assume $n$ is an integer.
  Prove by contraposition (explicitly state the contrapositive).
  \begin{enumerate}[(a)]
  \setcounter{enumi}{10}
  \item 3 divides $n-2\imp n$ is not a perfect square.
  \item If $p>2$ is prime, then $p^2+1$ is composite.
  \end{enumerate}

  \subsection*{(k) P(x) : (p)3 divides $n-2\imp n$ (q)is not a perfect square.}
  
  Claim : If 3 divides n - 2, then n is not a perfect square

  Contrapositive: Assume that n is a perfect square, then 3 does not divide n - 2

  If n is a perfect square, then n = \(k^2\) for some $k\in\Z$ .

  Now consider n - 2 in (3 divides n - 2), plug in \(k^2\) for n

  n - 2 = \( k^2\) - 2

  Now we want to show that 3 does not divide \(k^2\) - 2. In order for \(k^2\) to be divisble by 3, \((k^2-2)\%3\) has to equal 0. Consider 3 cases, k=0, k=1, and k=2.

  k = 0, \(k^2 = 0\), \(k^2-2=-2\), \(-2 \% 3 = 1\), remainder of 1 when divided by 3, not divisible by 3
  
  k = 1, \(k^2 = 1\), \(k^2-2=-1\), \(-1 \% 3 = 2\), remainder of 2 when divided by 3, not divisibly by 3

  k = 2, \(k^2 = 4\), \(k^2-2=2\), \(2 \% 3 = 1\), remainder of 1 when divided by 3, not divisible by 3

  In all cases, we see that \((k^2-2)\%3\) does not equal to 0, none of these cases are divisible by 3

  We have proven that when n is a perfect square, then 3 does not divide n - 2, making 3 divides n - 2 false

  Thus we have shown that p is false and P(x) is true.

  \subsection*{(l) P(x) : (p) If p $>$ 2 is prime, (q) then \(p^2+1\) is composite}
  
  Contrapositive: Assume that \(p^2+1\) is not composite, then p $>$ 2 is not prime

  Because we are assuming that $p^2 + 1$ is not composite, and that p $>$ 2, we can say that $p^2 + 1$ results in an odd number

  And because $p^2 + 1$ results in a odd number, we can assume that p is even. (Because an even number squared is another even number, add 1 and that makes it odd)

  We have proven that for all cases $p > 2$, $p^2 + 1$ results in an odd number. Proving p to be even number $>$ 2, which can never be a prime number because its divisible by 2 and thus a composite number.

  Thus we have proved p $>$ 2 to not be prime, making (p) false and P(x) true.

\vspace{0.1in}

\item \textbf{Problem 4.11.}
  For $x,y\in\N$, which statements below are contradictions (cannot possibly be true).
  Explain.
%  \begin{multicols}{2}
  \begin{enumerate}[(a)]
  \item $x^2<y$.
  \item $x^2=y/2$.
  \item $x^2-y^2\le 1$.
  \item $x^2+y^2\le 1$.
  \item $2x+1=y^2+5y$.
  \item $x^2-y^2/2=1$.
  \item $x^2-y^2=1$.
  \end{enumerate}
%  \end{multicols}

\vspace{0.1in}

\item \textbf{Problem 4.14.}
  Prove: If $a,b,c\in\Z$ are odd, then for all $x\in\Q$,
  $ax^2+bx+c\ne 0$.
  (Contradiction in a direct proof.)

\newpage

%\item \textbf{Problem 4.16.}

\item \textbf{Problem 4.47 (Without Loss of Generality (wlog)).}
  Consider the following claim.
  \begin{center}
  If $x$ and $y$ have opposite parity (one is odd and one is even),
  then $x+y$ is odd.
  \end{center}
  Explain why, in a direct proof, we may assume that $x$ is odd and $y$ is even?
  Prove the claim.
  (Such a proof starts ``Without loss of generality, assume $x$ is odd and $y$ is even.
  Then, $\ldots$'')

\vspace{0.1in}

\item \textbf{Problem 4.48(c).}
  Use the concept of ``without loss of generality'' to prove these claims.
  \begin{enumerate}[(a)]
  \setcounter{enumi}{2}
  \item For any non-zero real number $x$, $x^2+1/x^2\ge 2$.
  \end{enumerate}

\vspace{0.1in}

%\item \textbf{Problem 5.10.}

\item \textbf{*Problem 5.12(d).}
  For $n\ge 1$, prove by induction:
  \begin{enumerate}[(a)]
  \setcounter{enumi}{3}
  \item $3^n>n^2$.
  \end{enumerate}

  \subsection*{(d) P(n) : \(3^n > n^2\)}
  We use induction prove $\forall(n \geq 1)$ : P(n)
  
  [Base Case] $P(1) = 3^1 > 1^2 = 3 > 1$ which is True

  [induction Step] Prove $P(n) \imp P(n+1)$ for all n $\geq$ 1 

  Specifically, if $3^n > n^2$, then $3^{n+1} > (n+1)^2$

  Now we want to prove $3^{n+1} > (n+1)^2$

  By Induction Assumption, we know that $3^n > n^2$

  Which allows us to multiply both sides by 3, giving us $3^{n+1} > 3n^2$, looking at the LHS, we can subsitute it in our next inequality

  Simplify $(n+1)^2$ gives us $ n^2 + 2n +1$
  
  Subsitute everything into this next inequality so we can compare, we get

  $3^{n+1} > 3n^2 > n^2 + 2n + 1$

  We know that $3k^2$ is greater than the highest term in the most RHS's quadratic $n^2$ and that $3^{n+1} > 3n^2$ due to the induction hypothesis. 

  We can rewrite the inequality as $3^{n+1} > n^2 + 2n + 1$ which is equal/the same to what we wanted to prove, $3^{n+1} > (n+1)^2$, up in the 4th line
  
  Thus, by induction, we have proved P(n+1) to be true, making P(n) to be true for all cases $n \geq 1$

\vspace{0.1in}

\item \textbf{*Problem 5.20.}
  Prove, by induction, that every $n\ge 1$ is a sum of distinct powers of~2.

  [Base Case] : 1 = $2^0$, True

  [Induction Hypothesis] : n = $2^k$ n $\geq$ 1

  Consider these two cases

  Case 1: k + 1 is even

  1. $(k+1)/2$ must be a natural number less than or equal to k

  2. $(k+1)/2$ = $2^{x_1} + 2^{x_2} + . . . + 2^{x_y} $ for $0 \leq x_1 < ... < x_y$ 

  3. $k+1 = 2(2^{x_1} + 2^{x_2} + ... + 2^{x_y})$ ; multiply both sides by 2

  4. $k+1 = 2^{x_1+1} + ... + 2^{x_y+1}$ ; can be represented as a distinct sum of $2^{x_y}$
  
  Case 2: k + 1 is odd

  1. k must be even, thus each power much be atleast 1 so k is divisble by 2

  2. k = $2^{x_1} + ... +2^{x_y}$ for $0 < x_1 < x_y$

  3. k + 1 = $2^{x_1} + ... +2^{x_y} + 1 = 2^{x_1} + ... +2^{x_y} + 2^0$ ; can also be represented as a distinct sum of $2^{x_y}$

  Thus, by induction, we have shown that for every n $\geq$ 1 is a sum of a distinct powers of 2

\vspace{0.1in}

\item \textbf{*Problem 5.39.}
  Prove you can make any postage greater than 12\textcent\
  using only 4\textcent\  and 5\textcent\  stamps.
  (The USPS can set any postage above 12\textcent\  and you don't have to buy any new stamps.)

  [Base Cases] (up to 15 because 12 + 4 = 16 which repeats 12)

  12 = 4, 4, 4 , 13 = 4, 4, 5 , 14 = 4, 5, 5 , 15 = 5, 5, 5

  [Induction Step]

  P(n) $\imp$ P(n+1)

  P(n) = 4k + 5j

  P(n+4) = (4k + 5j) + 4

  P(n+4) = 4(k+1) + 5j

  Since we have shown that every for every four cents, you can add a four cent stamp and that there are four base cases, the induction proof shows that there is a combination of 4 and 5 cent stamps that can make of every value $ \geq $ 12.


\vspace{0.1in}

\item \textbf{Problem 6.3(a).}
  Strengthen the claim and prove by induction for $n\ge 1$:
  \begin{enumerate}[(a)]
  \item The sum of the first $n$ odd numbers is a square.
    [Hint: Strengthen to a specific square.]
  \end{enumerate}

\vspace{0.1in}

\item \textbf{*Problem 6.8.}
  Prove $n^7<2^n$ for $n\ge 37$.
  (a)~Use induction.
  (b)~Use leaping induction.

  \subsection*{a - Using induction}

  1.[Base Case] $37^7 < 2^{37}$

  94, 931, 877, 133 $<$ 137, 938, 953, 472 ; true

  2.[Induction Step] : P(n) $\imp$ P(n+1)

  3.P(n) = $n^7 < 2^n$

  4.P(n+1) = $(n+1)^7 < 2^{n+1}$

  5.$(n+1)^7 = n^7 + 7n^6 + 21n^5 + 35n^4 + 35n^3 + 21n^2 + 7n + 1$ ; $7 < n $ | $21 < n^2$ | $35 < n^3$

  6.$(n+1)^7 < n^7 + (n)n^6 + (n^2)n^5 + (n^3)n^4 + (n^4)n^3 + (n^5)n^2 + (n^6)n + n^7$

  7.$(n+1)^7 < 8n^7$

  8. $8n^7 < 2^{n+1}$

  Since the left side is linear but the right side is exponential, we have proven the induction to be true

  \subsection*{b - Using leaping induction}

  1.[Base Cases]

  For 37: 9.4 * $10^{10} < 1.3 * 10^{11}$

  For 38: 1.1 * $10^{11} < 2.7 * 10^{11}$

  2. P(n) = $n^7 < 2^n$

  3. P(n+2) = $(n+2)^7 < 2^{n+2}$

  \marginnote{\textcolor{blue}{\textbf{Note:}} $14<n$\\$84<n^2$\\$280<n^3$\\$560<n^4$\\$672<n^5$}[-2cm]

  4. $(n+2)^7 = n^7 + 14n^6 + 84n^5 + 280n^4 + 560n^3 + 672n^2 + 448n + 128$
  
  5. $(n+2)^7 < n^7 + (n)n^6 + (n^2)n^5 + (n^3)n^4 + (n^4)n^3 + (n^5)n^2 + (n^6)n + n^7$

  6. $(n+2)^7 < 8n^7$

  7. $8n^7 < 2^{n+2}$

  Since the left side is linear but the right side is exponential, we have proven the induction to be true.

\vspace{0.1in}

\item \textbf{Problem 6.15.}
  Prove that there are $2^{\lceil n/2\rceil}$ distinct $n$-bit binary palindromes
  (strings that equal their reversal).

\vspace{0.1in}

\item \textbf{Problem 6.32.}
  We are back in $L$-tile land.
  \begin{enumerate}[(a)]
  \item This time, the potted plant needs more room than just one square.
    For $n\ge 1$, a $2^n\times 2^n$ grid-patio is missing a (large) $2\times 2$ square
    in a corner as shown in the figure.
    Prove that the remainder of the patio can be $L$-tiled, for $n\ge 1$.
  \item We are no longer sure what the size of the potted plant is.
    The size may be $2^k\times 2^k$, and so a $2^k\times 2^k$ square
    will be missing from the corner of the $2^n\times 2^n$ grid-patio.
    Prove that the remainder of the patio can always be $L$-tiled,
    for $k\ge 1$ and $n\ge k$.
    [Hint: Tinker: try $k=2$; $n=3$ and $k=2$; $n=4$ to figure out what is going on.]
  \end{enumerate}

\newpage

\item \textbf{*Problem 6.43.}
  A sliding puzzle is a grid of 9 squares with 8 tiles.
  The goal is to get the 8 tiles into order (the target configuration).
  A move slides a tile into an empty square.
  Below, we show first a row move, then a column move.

  Prove that no sequence of moves produces the target configuration.
  [Hint: The tiles form a sequence going left to right, top to bottom.
  An inversion is a pair that is out of order.
  Prove by induction that the number of inversions stays odd.]

  1. [Base Case] Starting out, there is an odd number of inversions (i.e., (8,7) as the number 8 should be after 7)

  2. [Induction Hypothesis] Assume $Move_k$ is odd, prove $Move_{k+1}$ is odd

  3. [Induction Step] Let's consider two cases

  Case 1 : Row moves

  We can see that row moves always keeps the same number of inversions

  If we define the grid as a set of numbers ex. 
  
  \{1, 2, 3, 4, \textunderscore, 6, 8, 5, 7\}

  Here there are 3 inversions: (6,5), (8, 5), (8, 7)

  And no matter if we move the 4 to the right, or 6 to the left, the order stays the same
  and thus, so do the inversions.

  Case 2 : Column moves

  With column moves, there are 2 possiblities, either going up, or down

  Thinking of the grid as a set (i.e., \{1, 2, 3, 4, \textunderscore, 6, 8, 5, 7\})

  It makes it clear that these two possibilties both make the current number jump over two indexes in the set

  This can either be two directions up, or down, but the resulting inversion count changes, depends on the numbers involved 

  Because of this, there are 3 cases that we must consider:

  Case 2.1 : Increases inversion count by 2

  If you jump down and the two values you jump over are greater than the jumping value, or
  when you are jumping up, and the two values you jump over are less than the jumping value, then the inversion count will increase by two.

  Ex. \{1, \textunderscore, 3, 4, 2, 6, 8, 5, 7\} Inversions now include (3,2) (4,2)

  Case 2.2 : Decreases inversion count by 2

  This is the opposite of the above example
  
  If you jump up and the two values you jump over are greater than the jumping value, or
  when you are jumping down, and the two values you jump over are less than the jumping value, then the inversion count will decrease by two.

  Ex. if we revert the previous example
  
  \{1, 2, 3, 4, \textunderscore, 6, 8, 5, 7\} Inversions now do not include (3,2) (4,2)

  Case 2.3 : Inversion count remains the same

  If you jump either direction over 1 value that is greater, and 1 value that is less, the inversion count will remain the same

  [Conclusion] Thus, since the inversion count starts off odd, and the only possible moves involves either changing the value by 2 or keeping it the same, the value of the inversion count will always be odd. 

\vspace{0.1in}

\item \textbf{Problem 7.3.}
  Give a recursive definition of the function $f(n)=n!\times 2^n$, where $n\ge 1$.

\vspace{0.1in}

\item \textbf{Problem 7.4(a-c).}
  Guess a formula for $A_n$ and prove it by induction.
  \begin{enumerate}[(a)]
  \item $A_0=0$ and $A_n=A_{n-1}+1$ for $n\ge 1$.
  \item $A_1=1$, $A_2=2$, and $A_n=A_{n-1}+2A_{n-2}$ for $n\ge 2$.
  \item $A_0=1$; $A_1=2$; $A_n=2A_{n-1}-A_{n-2}+2$ for $n\ge 2$.
    [Hint: Method of differences.]
  \end{enumerate}

\end{itemize}

\end{document}
