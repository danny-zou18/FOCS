\documentclass[11pt]{article}
\usepackage{datetime}
\usepackage{color,array,graphics}
\usepackage{enumerate}
\usepackage[pdftex, colorlinks, linkcolor=red,citecolor=red,urlcolor=blue]{hyperref}
\usepackage{ulem}

\setlength{\parindent}{0cm}

\setlength{\parskip}{0.3cm plus4mm minus3mm}

\textwidth  6.5in
\oddsidemargin +0.0in
\evensidemargin +0.0in
\textheight 9.0in
\topmargin -0.5in

\usepackage{upquote,textcomp}
\usepackage{amssymb,amsmath,amsfonts,amsthm}
\usepackage{graphicx}
\usepackage{multicol}

\newcommand{\vect}[1]{{\bf #1}}                 %for bold chars
\newcommand{\vecg}[1]{\mbox{\boldmath $ #1 $}}  %for bold greek chars
\newcommand{\matx}[1]{{\bf #1}}

\def\OR{\vee}
\def\AND{\wedge}
\def\imp{\rightarrow}
\def\math#1{$#1$}

\DeclareSymbolFont{AMSb}{U}{msb}{m}{n}
\DeclareMathSymbol{\N}{\mathbin}{AMSb}{"4E}
\DeclareMathSymbol{\Z}{\mathbin}{AMSb}{"5A}
\DeclareMathSymbol{\R}{\mathbin}{AMSb}{"52}
\DeclareMathSymbol{\Q}{\mathbin}{AMSb}{"51}
\DeclareMathSymbol{\I}{\mathbin}{AMSb}{"49}
\DeclareMathSymbol{\C}{\mathbin}{AMSb}{"43}

\begin{document}
\thispagestyle{empty}

\begin{center}
\large
\textbf{CSCI 2200 --- Foundations of Computer Science (FoCS) \\
Homework 3 (document version 1.0)}
\end{center}

\section*{Overview}
\begin{itemize}
\item This homework is due by 11:59PM on Thursday, October~19
\item You may work on this homework in a group of up to four students;
  unlike recitation problem sets,
  \textbf{your teammates may be in any section}
\item You may use at most \textbf{two} late days on this assignment
\item Please start this homework early and ask questions during
  office hours and at your October~18 recitation section;
  also ask questions on the Discussion Forum
\item Please be concise in your written answers;
  even if your solution is correct, if it is not well-presented,
  you may still lose points
\item You can type or hand-write (or both) your solutions
  to the required graded problems below;
  \textbf{all work must be organized in one PDF that lists
  all teammate names}
\item You are strongly encouraged to use LaTeX, in particular for
  mathematical symbols;
  see the corresponding \verb+hw1.tex+ file as a starting point
  and example
\end{itemize}

%\vspace{0.2in}

\section*{Grading}
\begin{itemize}
\item For each assigned problem, a grade of 0, 1, or 2 is assigned
  as follows:
  0 indicates no credit;
  1 indicates half credit;
  and 2 indicates full credit
\item No credit is assigned if a problem is not attempted
  or minimal work/progress is shown
\item Half credit is assigned if a strong attempt was made
  toward a solution and/or only part of the problem was attempted or solved
\item Full credit is assigned for a perfect or nearly perfect solution,
  i.e.,~only one or two minor typos/mistakes at most
\end{itemize}

\vspace{0.2in}

\begin{center}
\end{center}

\newpage
\section*{Warm-up exercises}
The problems below are good practice problems to work on.
Do not submit these as part of your homework submission.
\textbf{These are ungraded problems.}

\begin{multicols}{2}
\begin{itemize}

\item \textbf{Problem 7.9.}
\item \textbf{Problem 7.11.}
\item \textbf{Problem 7.12(a-b).}\\(See Problem 7.28 for hints.)
%\item \textbf{Problem 7.19(k).}
\item \textbf{Problem 7.21.}
\item \textbf{Problem 7.41.}
\item \textbf{Problem 7.44.}
\item \textbf{Problem 7.45(a-b,d-f).}
\item \textbf{Problem 7.46.}
\item \textbf{Problem 7.47.}
\item \textbf{Problem 7.49.}
\item \textbf{Problem 8.12(a-c).}
\item \textbf{Problem 8.13.}
\item \textbf{Problem 8.18.}

\end{itemize}
\end{multicols}

\section*{Graded problems}
The problems below are required and will be graded.
\begin{itemize}

\item \textbf{*Problem 7.12(c).} (See Problem 7.28 for hints.)
\item \textbf{*Problem 7.13(a).}
\item \textbf{*Problem 7.19(d).}
\item \textbf{*Problem 7.42.}
\item \textbf{*Problem 7.45(c).}
\item \textbf{*Problem 8.12(d).}
\item \textbf{*Problem 8.14.}

\end{itemize}

As you might not have the required textbook yet,
all of the above problems (both graded and ungraded)
are transcribed in the pages that follow.

Graded problems are noted with an asterisk~(*).

If any typos exist below, please use the textbook description.

\newpage
\begin{itemize}

\item \textbf{Problem 7.9.}
$G_0=0$, $G_1=1$, and $G_n=7G_{n-1}-12G_{n-2}$ for $n>1$.
Compute $G_5$.

Show $G_n=4^n-3^n$ for $n\ge 0$.

\vspace{0.1in}

\item \textbf{Problem 7.11.}
In each case tinker.
Then, guess a formula that solves the recurrence, and prove it.
\begin{enumerate}[(a)]
\item $P_0=0$, $P_1=a$, and $P_n=2P_{n-1}-P_{n-2}$, for $n>1$.
\item $G_1=1$; $G_n=(1-1/n)\cdot G_{n-1}$, for $n>1$.
\end{enumerate}

\vspace{0.1in}

\item \textbf{Problem 7.12(a-b).} (See Problem 7.28 for hints.)
Tinker to guess a formula for each recurrence and prove it.
In each case, $A_1=1$ and for $n>1$:
\begin{enumerate}[(a)]
\item $A_n=10A_{n-1}+1$.
\item $A_n=nA_{n-1}/(n-1)+n$.
\end{enumerate}

\vspace{0.1in}

\item \textbf{*Problem 7.12(c).} (See Problem 7.28 for hints.)
Tinker to guess a formula for each recurrence and prove it.
In each case, $A_1=1$ and for $n>1$:
\begin{enumerate}[(a)]
\setcounter{enumi}{2}
\item $A_n=10nA_{n-1}/(n-1)+n$.
\end{enumerate}

\vspace{0.1in}

\item \textbf{*Problem 7.13(a).}
Analyze these very fast-growing recursions.
[Hint: Take logarithms.]
\begin{enumerate}[(a)]
\item $M_1=2$ and $M_n=aM_{n-1}^2$ for $n>1$.
Guess and prove a formula for $M_n$.
Tinker, tinker.
\end{enumerate}

\vspace{0.1in}

\item \textbf{*Problem 7.19(d).}
Recall the Fibonacci numbers: $F_1,F_2=1$; and, $F_n=F_{n-1}+F_{n-2}$ for $n>2$.
\begin{enumerate}[(a)]
\setcounter{enumi}{3}
\item Prove that every third Fibonacci number, $F_{3n}$, is even.
\end{enumerate}

\vspace{0.1in}

\item \textbf{Problem 7.21.}
Show that every $n\ge 1$ is a sum of distinct Fibonacci numbers,
e.g.,~$11=F_4+F_6$; $20=F_3=F_5+F_7$.
(There can be many ways to do it, e.g.,~$6=F_1+F_5=F_2+F_3+F_4$.)
[Hints: Greedy algorithm; strong induction.]

\vspace{0.1in}

\item \textbf{Problem 7.41.}
Refer to the pseudocode on the right.
\begin{verbatim}
out=S([arr],i,j)
 if(j<i) out=0;
 else
  out=arr[j]+S([arr],i,j-1);
\end{verbatim}
\begin{enumerate}[(a)]
\item What is the function being implemented?
\item Prove that the output is correct for every valid input.
\item Give a recurrence for the runtime $T_n$, where $n=j-i$.
\item Guess and prove a formula for $T_n$.
\end{enumerate}

\vspace{0.1in}

\item \textbf{*Problem 7.42.}
Give pseudocode for a recursive function that computes $3^{2^n}$ on input $n$.
\begin{enumerate}[(a)]
\item Prove that your function correctly computes $3^{2^n}$ for every $n\ge 0$.
\item Obtain a recurrence for the runtime $T_n$.
  Guess and prove a formula for $T_n$.
\end{enumerate}

\vspace{0.1in}

\item \textbf{Problem 7.44.}
We give two implementations of \verb+Big(n)+ from page~90
(\verb+iseven(n)+ tests if $n$ is even).
\begin{multicols}{2}
\begin{enumerate}[(a)]
\item
\begin{verbatim}
out=Big(n)
 if(n==0) out=1;
 elseif(iseven(n))
    out=Big(n/2)*Big(n/2);
 else out=2*Big(n-1)
\end{verbatim}
\item
\begin{verbatim}
out=Big(n)
 if(n==0) out=1;
 elseif(iseven(n))
    tmp=Big(n/2); out=tmp*tmp;
 else out=2*Big(n-1)
\end{verbatim}
\end{enumerate}
\end{multicols}

\begin{enumerate}[(i)]
\item For each, prove that the output is $2^n$ and give a recurrence for the runtime $T_n$.
  (\verb+iseven(n)+ is two operations.)
\item For each, compute runtimes $T_n$ for $n=1,\ldots,10$.
  Compare runtimes with Exercise~7.10 on page~90.
\end{enumerate}

\vspace{0.1in}

\item \textbf{Problem 7.45(a-b,d-f).}
Give recursive definitions for the set $\mathcal{S}$ in each of the following cases.
\begin{enumerate}[(a)]
\item $\mathcal{S}=\{0,3,6,9,12,\dots\}$, the multiples of~3.
\item $\mathcal{S}=\{1,2,3,4,6,7,8,9,11,\dots\}$, the numbers which are not multiples of~5.
\setcounter{enumi}{3}
\item The set of odd multiples of~3.
\item The set of binary strings with an even number of 0's.
\item The set of binary strings of even length.
\end{enumerate}

\vspace{0.1in}

\item \textbf{*Problem 7.45(c).}
Give recursive definitions for the set $\mathcal{S}$ in each of the following cases.
\begin{enumerate}[(a)]
\setcounter{enumi}{2}
\item $\mathcal{S}=\{$\ all strings with the same number of 0's and 1's\ $\}$
  (e.g.,~0011, 0101, 100101).
\end{enumerate}

\vspace{0.1in}

\item \textbf{Problem 7.46.}
What is the set $\mathcal{A}$ defined recursively as shown?
(By default, nothing else is in $\mathcal{A}$---minimality.)
\begin{enumerate}[(1)]
\item $1\in\mathcal{A}$
\item $x,y\in\mathcal{A}\imp x+y\in\mathcal{A}$ \\
  $x,y\in\mathcal{A}\imp x-y\in\mathcal{A}$
\end{enumerate}

\vspace{0.1in}

\item \textbf{Problem 7.47.}
What is the set $\mathcal{A}$ defined recursively as shown?
(By default, nothing else is in $\mathcal{A}$---minimality.)
\begin{enumerate}[(1)]
\item $3\in\mathcal{A}$
\item $x,y\in\mathcal{A}\imp x+y\in\mathcal{A}$ \\
  $x,y\in\mathcal{A}\imp x-y\in\mathcal{A}$
\end{enumerate}

\vspace{0.1in}

\item \textbf{Problem 7.49.}
There are 5 rooted binary trees (RBTs) with 3 nodes.
How many have 4~nodes?

\vspace{0.1in}

\item \textbf{Problem 8.12(a-c).}
A set $\mathcal{P}$ of parenthesis strings has a recursive definition (right).
\begin{enumerate}[(1)]
\item $\varepsilon\in\mathcal{P}$
\item $x\in\mathcal{P}\imp [x]\in\mathcal{P}$ \\
  $x,y\in\mathcal{P}\imp xy\in\mathcal{P}$
\end{enumerate}
\begin{enumerate}[(a)]
\item Determine if each string is in $\mathcal{P}$ and
  give a derivation if it is in $\mathcal{P}$. \\
  (i)~$[[[]]]][$\ \ \ \ (ii)~$[][[]][[]]$\ \ \ \ (iii)~$[[][][]$
\item Give two derivations of $[][][[]]$ whose steps are not a simple reordering of each other.
\item Prove by structural induction that every string in $\mathcal{P}$ has even length.
\end{enumerate}

\vspace{0.1in}

\item \textbf{*Problem 8.12(d).}
A set $\mathcal{P}$ of parenthesis strings has a recursive definition (right).
\begin{enumerate}[(1)]
\item $\varepsilon\in\mathcal{P}$
\item $x\in\mathcal{P}\imp [x]\in\mathcal{P}$ \\
  $x,y\in\mathcal{P}\imp xy\in\mathcal{P}$
\end{enumerate}
\begin{enumerate}[(a)]
\setcounter{enumi}{3}
\item Prove by structural induction that every string in $\mathcal{P}$ is balanced.
\end{enumerate}

\vspace{0.1in}

\item \textbf{Problem 8.13.}
Recursively define the binary strings that contain more 0's than 1's.
Prove:
\begin{enumerate}[(a)]
\item Every string in your set has more 0's than 1's.
\item Every string which has more 0's than 1's is in your set.
\end{enumerate}

\vspace{0.1in}

\item \textbf{*Problem 8.14.}
A set $\mathcal{A}$ is defined recursively as shown.
\begin{enumerate}[(1)]
\item $3\in\mathcal{A}$
\item $x,y\in\mathcal{A}\imp x+y\in\mathcal{A}$ \\
  $x,y\in\mathcal{A}\imp x-y\in\mathcal{A}$
\end{enumerate}
\begin{enumerate}[(a)]
\item Prove that every element of $\mathcal{A}$ is a multiple of~3.
\item Prove that every multiple of 3 is in $\mathcal{A}$.
\end{enumerate}

\vspace{0.1in}

\item \textbf{Problem 8.18.}
Recursively define rooted binary trees (RBTs) and rooted full binary trees (RFBTs).
\begin{enumerate}[(a)]
\item Give examples, with derivations, of RBTs and RFBTs with 5, 6, and 7 vertices.
\item Prove by structural induction that every RFBT has an odd number of vertices.
\end{enumerate}

\end{itemize}

\end{document}
