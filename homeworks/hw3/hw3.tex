\documentclass[11pt]{article}
\usepackage{datetime}
\usepackage{color,array,graphics}
\usepackage{enumerate}
\usepackage[pdftex, colorlinks, linkcolor=red,citecolor=red,urlcolor=blue]{hyperref}
\usepackage{ulem}

\setlength{\parindent}{0cm}

\setlength{\parskip}{0.3cm plus4mm minus3mm}

\textwidth  6.5in
\oddsidemargin +0.0in
\evensidemargin +0.0in
\textheight 9.0in
\topmargin -0.5in

\usepackage{upquote,textcomp}
\usepackage{amssymb,amsmath,amsfonts,amsthm}
\usepackage{graphicx}
\usepackage{multicol}

\newcommand{\vect}[1]{{\bf #1}}                 %for bold chars
\newcommand{\vecg}[1]{\mbox{\boldmath $ #1 $}}  %for bold greek chars
\newcommand{\matx}[1]{{\bf #1}}

\def\OR{\vee}
\def\AND{\wedge}
\def\imp{\rightarrow}
\def\math#1{$#1$}

\DeclareSymbolFont{AMSb}{U}{msb}{m}{n}
\DeclareMathSymbol{\N}{\mathbin}{AMSb}{"4E}
\DeclareMathSymbol{\Z}{\mathbin}{AMSb}{"5A}
\DeclareMathSymbol{\R}{\mathbin}{AMSb}{"52}
\DeclareMathSymbol{\Q}{\mathbin}{AMSb}{"51}
\DeclareMathSymbol{\I}{\mathbin}{AMSb}{"49}
\DeclareMathSymbol{\C}{\mathbin}{AMSb}{"43}

\begin{document}
\thispagestyle{empty}

Team Members: Danny Zou, Aaron Verkleeren

\begin{center}
\large
\textbf{CSCI 2200 --- Foundations of Computer Science (FoCS) \\
Homework 3 (document version 1.0)}
\end{center}

\section*{Overview}
\begin{itemize}
\item This homework is due by 11:59PM on Thursday, October~19
\item You may work on this homework in a group of up to four students;
  unlike recitation problem sets,
  \textbf{your teammates may be in any section}
\item You may use at most \textbf{two} late days on this assignment
\item Please start this homework early and ask questions during
  office hours and at your October~18 recitation section;
  also ask questions on the Discussion Forum
\item Please be concise in your written answers;
  even if your solution is correct, if it is not well-presented,
  you may still lose points
\item You can type or hand-write (or both) your solutions
  to the required graded problems below;
  \textbf{all work must be organized in one PDF that lists
  all teammate names}
\item You are strongly encouraged to use LaTeX, in particular for
  mathematical symbols;
  see the corresponding \verb+hw1.tex+ file as a starting point
  and example
\end{itemize}

%\vspace{0.2in}

\section*{Grading}
\begin{itemize}
\item For each assigned problem, a grade of 0, 1, or 2 is assigned
  as follows:
  0 indicates no credit;
  1 indicates half credit;
  and 2 indicates full credit
\item No credit is assigned if a problem is not attempted
  or minimal work/progress is shown
\item Half credit is assigned if a strong attempt was made
  toward a solution and/or only part of the problem was attempted or solved
\item Full credit is assigned for a perfect or nearly perfect solution,
  i.e.,~only one or two minor typos/mistakes at most
\end{itemize}

\vspace{0.2in}

\begin{center}
\end{center}

\newpage
\section*{Warm-up exercises}
The problems below are good practice problems to work on.
Do not submit these as part of your homework submission.
\textbf{These are ungraded problems.}

\begin{multicols}{2}
\begin{itemize}

\item \textbf{Problem 7.9.}
\item \textbf{Problem 7.11.}
\item \textbf{Problem 7.12(a-b).}\\(See Problem 7.28 for hints.)
%\item \textbf{Problem 7.19(k).}
\item \textbf{Problem 7.21.}
\item \textbf{Problem 7.41.}
\item \textbf{Problem 7.44.}
\item \textbf{Problem 7.45(a-b,d-f).}
\item \textbf{Problem 7.46.}
\item \textbf{Problem 7.47.}
\item \textbf{Problem 7.49.}
\item \textbf{Problem 8.12(a-c).}
\item \textbf{Problem 8.13.}
\item \textbf{Problem 8.18.}

\end{itemize}
\end{multicols}

\section*{Graded problems}
The problems below are required and will be graded.
\begin{itemize}

\item \textbf{*Problem 7.12(c).} (See Problem 7.28 for hints.)
\item \textbf{*Problem 7.13(a).}
\item \textbf{*Problem 7.19(d).}
\item \textbf{*Problem 7.42.}
\item \textbf{*Problem 7.45(c).}
\item \textbf{*Problem 8.12(d).}
\item \textbf{*Problem 8.14.}

\end{itemize}

As you might not have the required textbook yet,
all of the above problems (both graded and ungraded)
are transcribed in the pages that follow.

Graded problems are noted with an asterisk~(*).

If any typos exist below, please use the textbook description.

\newpage
\begin{itemize}

\item \textbf{Problem 7.9.}
$G_0=0$, $G_1=1$, and $G_n=7G_{n-1}-12G_{n-2}$ for $n>1$.
Compute $G_5$.

Show $G_n=4^n-3^n$ for $n\ge 0$.

\vspace{0.1in}

\item \textbf{Problem 7.11.}
In each case tinker.
Then, guess a formula that solves the recurrence, and prove it.
\begin{enumerate}[(a)]
\item $P_0=0$, $P_1=a$, and $P_n=2P_{n-1}-P_{n-2}$, for $n>1$.
\item $G_1=1$; $G_n=(1-1/n)\cdot G_{n-1}$, for $n>1$.
\end{enumerate}

\vspace{0.1in}

\item \textbf{Problem 7.12(a-b).} (See Problem 7.28 for hints.)
Tinker to guess a formula for each recurrence and prove it.
In each case, $A_1=1$ and for $n>1$:
\begin{enumerate}[(a)]
\item $A_n=10A_{n-1}+1$.
\item $A_n=nA_{n-1}/(n-1)+n$.
\end{enumerate}

\vspace{0.1in}

\item \textbf{*Problem 7.12(c).} (See Problem 7.28 for hints.)
Tinker to guess a formula for each recurrence and prove it.
In each case, $A_1=1$ and for $n>1$:
\begin{enumerate}[(a)]
\setcounter{enumi}{2}
\item $A_n=10nA_{n-1}/(n-1)+n$.
\end{enumerate}

$ A_n = 
\begin{cases}
  1 & \text{if } n = 1 \\
  \frac{10nA_{n-1}}{(n-1)} + n & \text{if } n > 1
\end{cases}
$

$A_2 = \frac{20(1)}{1} + 2 = 22 = 2(10^1 + 1)$

$A_3 = \frac{30(22)}{2} + 3 = 333 = 3(10^2 + 10^1+ 1)$

$A_4 = \frac{40(333)}{3} + 4 = 4444 = 4(10^3 + 10^2 + 10^1 + 1)$

$A_5 = \frac{50(4444)}{4} + 5 = 55555 = 5(10^4 + 10^3 + 10^2 + 10^1 + 1)$

And so on\dots

Observe that A(n) = $n(10^{n-1} + 10^{n-2} + ... + 10^0)$

Define claim P(n) : A(n) = $n(10^{n-1} + 10^{n-2} + ...+ 10^0)$ for all n $>$ 1

We prove by strong induction that P(n) is true for n $>$ 1

[Base Case] A(1) = $1(10^{1-1}) = 1(1) = 1$ $A_1 = 1$ --- 1=1 True

[Induction Step] We prove $P(1) \land\ P(2) \land\ ... \land\ P(n) \imp P(n+1) $ for n $>$ 1

We must prove that $A_{n+1}$= $(n + 1)(10^{n} + 10^{n-1} + ... + 10^0)$

LHS : $A_{n+1} = \frac{10(n+1)}{n}(A_n)+ (n + 1)$

$A_{n+1} = \frac{(10(n+1))(n(10^{n-1} + 10^{n-2} + ...+ 10^0))}{n} + n + 1$

$A_{n+1} = (10(n+1))(10^{n-1} + 10^{n-2} + ...+ 10^0) + n + 1$

$A_{n+1} = (n+1)(10^{n} + 10^{n-1} + ...+ 10^0) + n + 1$

$A_{n+1} = (n+1)[(10^{n} + 10^{n-1} + ...+ 10^1) + 1]$

$A_{n+1} = (n+1)(10^{n} + 10^{n-1} + ...+ 10^1 + 1)$

$A_{n+1} = (n+1)(10^{n} + 10^{n-1} + ...+ 10^1 + 10^0)$

$A_{n+1} = (n+1)(10^{n} + 10^{n-1} + ...+ 10^0)$

Thus, $A_{n+1} = (n+1)(10^{n} + 10^{n-1} + ...+ 10^0)$, as was to be shown

By induction, P(n) is true for all n $>$ 1.

\vspace{0.1in}

\item \textbf{*Problem 7.13(a).}
Analyze these very fast-growing recursions.
[Hint: Take logarithms.]
\begin{enumerate}[(a)]
\item $M_1=2$ and $M_n=aM_{n-1}^2$ for $n>1$.
Guess and prove a formula for $M_n$.
Tinker, tinker.
\end{enumerate}

$ M_n = 
\begin{cases}
  2 & \text{if } n = 1 \\
  aM_{n-1}^2 & \text{if } n > 1
\end{cases}
$

$M_2 = a(M_1^2) = a(2^2) = a(2^{2^1})$

$M_3 = a(M_2^2) = a((4a)^2) = 16a^3 = 2^4(a^3) = 2^{2^2}(a^3)$

$M_4 = a(M_3^2) = a((16a^3)^2) = 256a^7 = 2^8(a^7) = 2^{2^3}(a^7)$

$M_5 = a(M_4^2) = a((256a^7))^2 = 65536a^15 = 2^{16}(a^{15}) = 2^{2^4}(a^{15})$

And so on \dots

We observe that M(n) = $2^{2^{n-1}} * a^{2^{n-1} - 1} $

Define claim P(n) : M(n) = $2^{2^{n-1}} * a^{2^{n-1} - 1} $ for all n $>$ 1

We prove by strong induction that P(n) is true for n $>$ 1

[Base Case] M(1) = $2^1 * a^{1-1} = 2 $ $M_1 = 2$ --- 2=2 True

[Induction Step] We prove $P(1) \land P(2) \land ... \land P(n) \imp P(n+1) for n > 1 $

We must prove that $M_{n+1} = 2^{2^{n}} * a^{2^{n} - 1}$

LHS: $M_{n+1} = aM_{n}^2$

$M_{n+1} = a(2^{2^{n-1}}*a^{2^{n-1}-1})^2$

$M_{n+1} = a(2^{2 * 2^{n-1}}*a^{2 * (2^{n-1}-1)})$

$M_{n+1} = a(2^{2^n} * a^{2^n - 2})$

$M_{n+1} = 2^{2^n} * a^{2^n-1}$

Thus, $M_{n+1} = 2^{2^n} * a^{2^n-1}$, as was to be shown

By induction, P(n) is true for all n $>$ 1.

\vspace{0.1in}

\item \textbf{*Problem 7.19(d).}
Recall the Fibonacci numbers: $F_1,F_2=1$; and, $F_n=F_{n-1}+F_{n-2}$ for $n>2$.
\begin{enumerate}[(a)]
\setcounter{enumi}{3}
\item Prove that every third Fibonacci number, $F_{3n}$, is even.
\end{enumerate}

$ F_n = 
\begin{cases}
  1 & \text{if } n = 1 \\
  1 & \text{if } n = 2 \\
  F_{n-1} + F_{n-2} & \text{if } n > 2
\end{cases}
$

\vspace*{0.1in}

Define claim P(n), where $F_{3n}$ is even

P(n) : $F_{3n}$ is even

Now we want to prove P(n) to be true for all n $\geq$ 1

Proof by Induction

[Base Case] For n = 1, $F_1 = 1 \ F_2 = 1 \ F_3 = F_1 + F_2 \ F_3 = 1 + 1 = 2$ which is even, True 

[Induction Step] Assume that P(n) is true, then we must prove that P(n) $\imp$ P(n+1) for all n $>$ 1

P(n+1) = $F_{3(n+1)} = F_{3n + 3}$

Plug into $F_n = F_{n-1} + F_{n-2}$

$F_{3n+3} = F_{3n+2} + F_{3n + 1}$

$F_{3n+2} = F_{3n+1} + F{3n}$

Subsitute, we get $F_{3n+3} = F_{3n+1} + F_{3n+1} + F_{3n}$

$F_{3n+3} = 2 * F_{3n+1} + F_{3n}$

We know from our induction hypothesis that $F_{3n}$ is even.

$2*F_{3n+1}$ is also even because it is a multiple of 2.

The sum of two even numbers will always come out to be even, thus proving 

$F_{3n+3} = 2 * F_{3n+1} + F_{3n}$ to be even

Thus, we have proven our claim P(n) to be true for all n $>$ 1, proving that every third Fibonacci number $F_{3n}$ is even.

\vspace{0.1in}

\item \textbf{Problem 7.21.}
Show that every $n\ge 1$ is a sum of distinct Fibonacci numbers,
e.g.,~$11=F_4+F_6$; $20=F_3=F_5+F_7$.
(There can be many ways to do it, e.g.,~$6=F_1+F_5=F_2+F_3+F_4$.)
[Hints: Greedy algorithm; strong induction.]

\vspace{0.1in}

\item \textbf{Problem 7.41.}
Refer to the pseudocode on the right.
\begin{verbatim}
out=S([arr],i,j)
 if(j<i) out=0;
 else
  out=arr[j]+S([arr],i,j-1);
\end{verbatim}
\begin{enumerate}[(a)]
\item What is the function being implemented?
\item Prove that the output is correct for every valid input.
\item Give a recurrence for the runtime $T_n$, where $n=j-i$.
\item Guess and prove a formula for $T_n$.
\end{enumerate}

\vspace{0.1in}

\item \textbf{*Problem 7.42.}
Give pseudocode for a recursive function that computes $3^{2^n}$ on input $n$.
\begin{enumerate}[(a)]
\item Prove that your function correctly computes $3^{2^n}$ for every $n\ge 0$.
\item Obtain a recurrence for the runtime $T_n$.
  Guess and prove a formula for $T_n$.
\end{enumerate}

\begin{equation}
  f(n) =
  \begin{cases}
      3 & \text{if } n = 0 \\
      f(n-1)^2 & \text{if } n \geq 1  
  \end{cases}
  \end{equation}
  \begin{enumerate}[(a)]
  \item Prove that your function correctly computes $3^{2^n}$ for every $n\ge 0$.
  
  \begin{center}
      Structural Induction Proof:
  \end{center}
  
  Base case:
  f(0) = 3 = $3^{2^{0}}$
  
  Induction hypothesis: $3^{2^{n}} = f(n)$, $f(n)=f(n-1)^2$
  
  1. $3^{2^{n+1}} = f(n+1)$; need to prove this
  
  2. $3^{2^{n+1}} = f(n)^2$; substitute the value of f(n+1) based on recursive definition
  
  3. $3^{2^{n+1}} = (3^{2^{n}})^2$; substitute in induction hypothesis
  
  4. $3^{2^{n+1}} = 3^{2^{n}2}$; algebraic manipulation
  
  5. $3^{2^{n+1}} = 3^{2^{n+1}}$; since both side are equal, we can conclude that the statement is true
  
  
  
  \vspace{0.3in}
  
  \item Obtain a recurrence for the runtime $T_n$.
    Guess and prove a formula for $T_n$.
  
  \begin{center}
      Formula for runtime: T$_{0}$ = 1, T$_{n}$ = T$_{n-1}$ + 1
  \end{center}
  
  Claim: The runtime of the function $T_n$ can be expressed as $T_n$ = O(n)
  
  Base case:
  
  1. T$_{0}$ = O(1), which is a constant time.
  
  Induction hypothesis:
  
  2. Assume that T$_{n}$ = O(n) for some integer n $\geq$ 0
  
  Induction steps:
  
  3. T$_{n+1}$ = T$_{n}$ + 1; need to prove this claim
  
  4. T$_{n+1}$ = O(n) + O(1); substitute in induction hypothesis
  
  5. T$_{n+1}$ = O(n+1); claim proved successfully
  
  Since we proved that increasing n by 1 will increase the runtime by 1, this means that the formula is proved true.
  
    
  \end{enumerate}
  

\vspace{0.1in}

\item \textbf{Problem 7.44.}
We give two implementations of \verb+Big(n)+ from page~90
(\verb+iseven(n)+ tests if $n$ is even).
\begin{multicols}{2}
\begin{enumerate}[(a)]
\item
\begin{verbatim}
out=Big(n)
 if(n==0) out=1;
 elseif(iseven(n))
    out=Big(n/2)*Big(n/2);
 else out=2*Big(n-1)
\end{verbatim}
\item
\begin{verbatim}
out=Big(n)
 if(n==0) out=1;
 elseif(iseven(n))
    tmp=Big(n/2); out=tmp*tmp;
 else out=2*Big(n-1)
\end{verbatim}
\end{enumerate}
\end{multicols}

\begin{enumerate}[(i)]
\item For each, prove that the output is $2^n$ and give a recurrence for the runtime $T_n$.
  (\verb+iseven(n)+ is two operations.)
\item For each, compute runtimes $T_n$ for $n=1,\ldots,10$.
  Compare runtimes with Exercise~7.10 on page~90.
\end{enumerate}

\vspace{0.1in}

\item \textbf{Problem 7.45(a-b,d-f).}
Give recursive definitions for the set $\mathcal{S}$ in each of the following cases.
\begin{enumerate}[(a)]
\item $\mathcal{S}=\{0,3,6,9,12,\dots\}$, the multiples of~3.
\item $\mathcal{S}=\{1,2,3,4,6,7,8,9,11,\dots\}$, the numbers which are not multiples of~5.
\setcounter{enumi}{3}
\item The set of odd multiples of~3.
\item The set of binary strings with an even number of 0's.
\item The set of binary strings of even length.
\end{enumerate}

\vspace{0.1in}

\item \textbf{*Problem 7.45(c).}
Give recursive definitions for the set $\mathcal{S}$ in each of the following cases.
\begin{enumerate}[(a)]
\setcounter{enumi}{2}
\item $\mathcal{S}=\{$\ all strings with the same number of 0's and 1's\ $\}$
  (e.g.,~0011, 0101, 100101).
\end{enumerate}

\begin{center}
  Recursive Definition for $\mathcal{S}$:
\end{center}

Base Cases:

1. $\varepsilon \in \mathcal{S}$; (i.e., the empty string is in $\mathcal{S}$)

Recursive Rules:

2. $x \in \mathcal{S} \xrightarrow{} x \cdot 01 \in \mathcal{S}$ and $x \cdot 10 \in \mathcal{S}$; In other words, for any string $x$ in $\mathcal{S}$, you can create new strings in $\mathcal{S}$ by appending either "01" or "10" to the end of $x$.

3. $x \in \mathcal{S} \xrightarrow{} 01 \cdot x \in \mathcal{S}$ and $10 \cdot x \in \mathcal{S}$; In other words, for any string $x$ in $\mathcal{S}$, you can create new strings in $\mathcal{S}$ by adding "01" or "10" to the beginning of $x$.

4. $x \in \mathcal{S} \xrightarrow{} 0 \cdot x \cdot 1 \in \mathcal{S}$ and $1 \cdot x \cdot 0 \in \mathcal{S}$; In other words, for any string $x$ in $\mathcal{S}$, you can create new strings in $\mathcal{S}$ by adding "0" or "1" to the start and then the opposite number to the end of $x$.

\vspace{0.1in}

\item \textbf{Problem 7.46.}
What is the set $\mathcal{A}$ defined recursively as shown?
(By default, nothing else is in $\mathcal{A}$---minimality.)
\begin{enumerate}[(1)]
\item $1\in\mathcal{A}$
\item $x,y\in\mathcal{A}\imp x+y\in\mathcal{A}$ \\
  $x,y\in\mathcal{A}\imp x-y\in\mathcal{A}$
\end{enumerate}

\vspace{0.1in}

\item \textbf{Problem 7.47.}
What is the set $\mathcal{A}$ defined recursively as shown?
(By default, nothing else is in $\mathcal{A}$---minimality.)
\begin{enumerate}[(1)]
\item $3\in\mathcal{A}$
\item $x,y\in\mathcal{A}\imp x+y\in\mathcal{A}$ \\
  $x,y\in\mathcal{A}\imp x-y\in\mathcal{A}$
\end{enumerate}

\vspace{0.1in}

\item \textbf{Problem 7.49.}
There are 5 rooted binary trees (RBTs) with 3 nodes.
How many have 4~nodes?

\vspace{0.1in}

\item \textbf{Problem 8.12(a-c).}
A set $\mathcal{P}$ of parenthesis strings has a recursive definition (right).
\begin{enumerate}[(1)]
\item $\varepsilon\in\mathcal{P}$
\item $x\in\mathcal{P}\imp [x]\in\mathcal{P}$ \\
  $x,y\in\mathcal{P}\imp xy\in\mathcal{P}$
\end{enumerate}
\begin{enumerate}[(a)]
\item Determine if each string is in $\mathcal{P}$ and
  give a derivation if it is in $\mathcal{P}$. \\
  (i)~$[[[]]]][$\ \ \ \ (ii)~$[][[]][[]]$\ \ \ \ (iii)~$[[][][]$
\item Give two derivations of $[][][[]]$ whose steps are not a simple reordering of each other.
\item Prove by structural induction that every string in $\mathcal{P}$ has even length.
\end{enumerate}

\vspace{0.1in}

\item \textbf{*Problem 8.12(d).}
A set $\mathcal{P}$ of parenthesis strings has a recursive definition (right).
\begin{enumerate}[(1)]
\item $\varepsilon\in\mathcal{P}$
\item $x\in\mathcal{P}\imp [x]\in\mathcal{P}$ \\
  $x,y\in\mathcal{P}\imp xy\in\mathcal{P}$
\end{enumerate}
\begin{enumerate}[(a)]
\setcounter{enumi}{3}
\item Prove by structural induction that every string in $\mathcal{P}$ is balanced.
\end{enumerate}

\begin{center}
  Structural Induction proof for $\mathcal{P}$:
\end{center}

Base case:

1. $\varepsilon$ has 0 open parenthesis and 0 close parenthesis, meaning it is balanced

2.  [ ] has 1 open parenthesis and 1 close parenthesis, meaning it is balanced

Structural Induction Steps: $x, y \in \mathcal{P} $, x and y are balanced

Case 1: first recursive step
\begin{quote}
3. $x \in \mathcal{P} \xrightarrow{} [x] \in \mathcal{P}$

4. $x = k(open) + k(close)$; rewrite x

5. $[x] = [k(open) + k(close)]$; add [] around both

6. $[x] = (k+1)(open) + (k+1)(close)$; since there are equal amounts of open and close, we know that [x] is balanced
\end{quote}

Case 2: second recursive step
\begin{quote}
7. $x, y \in \mathcal{P} \xrightarrow{} xy \in \mathcal{P}$

8. $x = k(open) + k(close), y = n(open) + n(close)$; since we know that x and y are both balanced, we can rewrite x and y

9. $xy = k(open) + k(close) + n(open) + n(close)$; we can rewrite xy as the concatenation of x and y

10. $xy = (k+n)(open) + (k+n)(close)$; since there are equal amounts of open and close, we know that xy is balanced 
\end{quote}

Since both recursive steps result in balanced strings, we can conclude that every string in $\mathcal{P}$ is balanced.

\vspace{0.1in}

\item \textbf{Problem 8.13.}
Recursively define the binary strings that contain more 0's than 1's.
Prove:
\begin{enumerate}[(a)]
\item Every string in your set has more 0's than 1's.
\item Every string which has more 0's than 1's is in your set.
\end{enumerate}

\vspace{0.1in}

\item \textbf{*Problem 8.14.}
A set $\mathcal{A}$ is defined recursively as shown.
\begin{enumerate}[(1)]
\item $3\in\mathcal{A}$
\item $x,y\in\mathcal{A}\imp x+y\in\mathcal{A}$ \\
  $x,y\in\mathcal{A}\imp x-y\in\mathcal{A}$
\end{enumerate}
\begin{enumerate}[(a)]
\item Prove that every element of $\mathcal{A}$ is a multiple of~3.

Base case:

1. 3 = 3*1; 3 is a multiple of 3

Induction steps: $x, y \in \mathcal{A}$, $x, y$ are multiples of 3

Case 1: first recursive step
\begin{quote}
2. $x,y \in \mathcal{A} \xrightarrow{} x+y \in \mathcal{A}$;

3. $x = 3k, y = 3n$; since both are multiples of 3 they can be represented like this with k and n being integers

4. $x+y = 3k + 3n$

5. $x+y = 3(k+n)$; here we can clearly see that x+y still results in a number that is a multiple of 3 

\end{quote}

Case 2: first recursive step
\begin{quote}
6. $x,y \in \mathcal{A} \xrightarrow{} x-y \in \mathcal{A}$;

7. $x = 3k, y = 3n$; same as above

8. $x-y = 3k - 3n$

9. $x-y = 3(k-n)$; here we can clearly see that x-y still results in a number that is a multiple of 3 

\end{quote}

Since both recursive steps result in a multiple of 3, we can conclude that every number in A must be a multiple of 3.

\vspace{0.2in}
\item Prove that every multiple of 3 is in $\mathcal{A}$.

Base Case:

1. 3 = 3*1; 3 is in $\mathcal{A} $.

Case 1: all multiples of 3 greater than 3 are in $\mathcal{A}$.

\begin{quote}
2. $x,y \in \mathcal{A} \xrightarrow{} x+y \in \mathcal{A}$; recursive statement 1, but we can rewrite this to help easily prove the above claim.

3. $y = 3, x = 3k$; rewrite x as some unknown multiple of 3, and y as the base case 3

4. $x+y = 3k + 3$; rewrite x+y to this function

5. $x+y = 3(k+1)$; this shows that all multiples of 3, greater than 3 are in $\mathcal{A}$.
\end{quote}

Case 2: all multiples of 3 less than 3 are in $\mathcal{A}$.

\begin{quote}
6. $x,y \in \mathcal{A} \xrightarrow{} x-y \in \mathcal{A}$; recursive statement 1, but we can rewrite this to help easily prove the above claim.

7. $y = 3, x = 3k$; rewrite x as some unknown multiple of 3, and y as the base case 3

8. $x+y = 3k - 3$; rewrite x+y to this function

9. $x+y = 3(k-1)$; this shows that all multiples of 3, less than 3 are in $\mathcal{A}$.
\end{quote}

Since we have proven that all multiples of 3 less than, greater than, and equal to 3 are in $\mathcal{A}$, we have proven that all multiples of 3 are in $\mathcal{A}$ 

\end{enumerate}

\vspace{0.1in}

\item \textbf{Problem 8.18.}
Recursively define rooted binary trees (RBTs) and rooted full binary trees (RFBTs).
\begin{enumerate}[(a)]
\item Give examples, with derivations, of RBTs and RFBTs with 5, 6, and 7 vertices.
\item Prove by structural induction that every RFBT has an odd number of vertices.
\end{enumerate}

\end{itemize}

\end{document}
