\documentclass[11pt]{article}
\usepackage{datetime}
\usepackage{color,array,graphics}
\usepackage{enumerate}
\usepackage[pdftex, colorlinks, linkcolor=red,citecolor=red,urlcolor=blue]{hyperref}
\usepackage{ulem}

\setlength{\parindent}{0cm}

\setlength{\parskip}{0.3cm plus4mm minus3mm}

\textwidth  6.5in
\oddsidemargin +0.0in
\evensidemargin +0.0in
\textheight 9.0in
\topmargin -0.5in

\usepackage{upquote,textcomp}
\usepackage{amssymb,amsmath,amsfonts,amsthm}
\usepackage{graphicx}
\usepackage{multicol}

\newcommand{\vect}[1]{{\bf #1}}                 %for bold chars
\newcommand{\vecg}[1]{\mbox{\boldmath $ #1 $}}  %for bold greek chars
\newcommand{\matx}[1]{{\bf #1}}

\def\OR{\vee}
\def\AND{\wedge}
\def\imp{\rightarrow}
\def\math#1{$#1$}

\DeclareSymbolFont{AMSb}{U}{msb}{m}{n}
\DeclareMathSymbol{\N}{\mathbin}{AMSb}{"4E}
\DeclareMathSymbol{\Z}{\mathbin}{AMSb}{"5A}
\DeclareMathSymbol{\R}{\mathbin}{AMSb}{"52}
\DeclareMathSymbol{\Q}{\mathbin}{AMSb}{"51}
\DeclareMathSymbol{\I}{\mathbin}{AMSb}{"49}
\DeclareMathSymbol{\C}{\mathbin}{AMSb}{"43}

\begin{document}
\thispagestyle{empty}

\begin{center}
\large
\textbf{CSCI 2200 --- Foundations of Computer Science (FoCS) \\
Homework 4 (document version 1.1)}
\end{center}

\section*{Overview}
\begin{itemize}
\item This homework is due by 11:59PM on Thursday, November~2
\item You may work on this homework in a group of up to four students;
  unlike recitation problem sets,
  \textbf{your teammates may be in any section}
\item You may use at most \textbf{two} late days on this assignment
\item Please start this homework early and ask questions during
  office hours and at your November~1 recitation section;
  also ask questions on the Discussion Forum
\item Please be concise in your written answers;
  even if your solution is correct, if it is not well-presented,
  you may still lose points
\item You can type or hand-write (or both) your solutions
  to the required graded problems below;
  \textbf{all work must be organized in one PDF that lists
  all teammate names}
\item You are strongly encouraged to use LaTeX, in particular for
  mathematical symbols;
  see the corresponding \verb+hw1.tex+ file as a starting point
  and example
\end{itemize}

%\vspace{0.2in}

\section*{Grading}
\begin{itemize}
\item For each assigned problem, a grade of 0, 1, or 2 is assigned
  as follows:
  0 indicates no credit;
  1 indicates half credit;
  and 2 indicates full credit
\item No credit is assigned if a problem is not attempted
  or minimal work/progress is shown
\item Half credit is assigned if a strong attempt was made
  toward a solution and/or only part of the problem was attempted or solved
\item Full credit is assigned for a perfect or nearly perfect solution,
  i.e.,~only one or two minor typos/mistakes at most
\end{itemize}

\vspace{0.2in}

%\begin{center}
%\includegraphics[scale=0.4]{math-bitmoji.png}
%\end{center}

\newpage
\section*{Warm-up exercises}
The problems below are good practice problems to work on.
Do not submit these as part of your homework submission.
\textbf{These are ungraded problems.}

\begin{multicols}{2}
\begin{itemize}

\item \textbf{Problem 8.12(a-c).}
\item \textbf{Problem 9.8(a).}
\item \textbf{Problem 9.20.}
\item \textbf{Problem 9.23(a).}
\item \textbf{Problem 9.28.}
\item \textbf{Problem 9.43.}
\item \textbf{Problem 9.69-9.72.}
\item \textbf{Problem 11.3.}
\item \textbf{Problem 11.10.}
\item \textbf{Problem 11.13.}
\item \textbf{Problem 11.15(a,c-d).}

\end{itemize}
\end{multicols}

\section*{Graded problems}
The problems below are required and will be graded.
\begin{itemize}

\item \textbf{*Problem 9.23(b).}
\item \textbf{*Problem 9.37. (v1.1)~See correction below.}
\item \textbf{*Problem 9.39.}
\item \textbf{*Problems 11.5 and 11.15(b).}
\item \textbf{*Problem 11.11.}
\item \textbf{*Problem 11.17.}
\item \textbf{*Problem 11.27.}

\end{itemize}

As you might not have the required textbook yet,
all of the above problems (both graded and ungraded)
are transcribed in the pages that follow.

Graded problems are noted with an asterisk~(*).

If any typos exist below, please use the textbook description.

\newpage
\begin{itemize}

\item \textbf{Problem 8.12(a-c).}
A set $\mathcal{P}$ of parenthesis strings has a recursive definition (right).
\begin{enumerate}[(1)]
\item $\varepsilon\in\mathcal{P}$
\item $x\in\mathcal{P}\imp [x]\in\mathcal{P}$ \\
  $x,y\in\mathcal{P}\imp xy\in\mathcal{P}$
\end{enumerate}
\begin{enumerate}[(a)]
\item Determine if each string is in $\mathcal{P}$ and
  give a derivation if it is in $\mathcal{P}$. \\
  (i)~$[[[]]]][$\ \ \ \ (ii)~$[][[]][[]]$\ \ \ \ (iii)~$[[][][]$
\item Give two derivations of $[][][[]]$ whose steps are not a simple reordering of each other.
\item Prove by structural induction that every string in $\mathcal{P}$ has even length.
\end{enumerate}

\vspace{0.1in}

\item \textbf{Problem 9.8(a).}
Estimate these sums.
\begin{enumerate}[(a)]
\item $\displaystyle \sum_{i=1}^{10} \sum_{j=1}^{20} 2^{i+j}$
\end{enumerate}

\vspace{0.1in}

\item \textbf{Problem 9.20.}
Prove or disprove:
\begin{enumerate}[(a)]
%\begin{multicols}{4}
\item $\displaystyle\frac{n^3+2n}{n^2+1}\in\Theta(n)$
\item $(n+1)!\in\Theta(n!)$
\item $n^{1/n}\in\Theta(1)$
\item $(n!)^{1/n}\in\Theta(n)$
%\end{multicols}
\end{enumerate}

\vspace{0.1in}

\item \textbf{Problem 9.23(a).}
Prove by contradiction: (a)~$n^3\not\in O(n^2)$

\vspace{0.1in}

\item \textbf{*Problem 9.23(b).}
Prove by contradiction: (b)~$2^n\not\in\Theta(3^n)$

\vspace*{0.15in}

Let us assume $2^n \in\Theta(3^n)$

By definition, if a function is in $\Theta(3^n)$, it means that it is both in $\Omega(3^n)$(Lower Bound) and $O(3^n)$(Upper Bound)

From this, we can define $2^n \in \Omega(3^n)$ and $2^n \in O(3^n)$ 

From the lower bound ($\Omega(3^n)$), there exists a $n_0 \in \mathbb{N}$ and a constant c $>$ 0 such that 

$c * 3^n \leq 2^n \: \: \forall \: \: n \geq n_0 $, now we can simplify

$c \leq \frac{2^n}{3^n} \: \: \forall \: \: n \geq 0$

to get $c \leq (\frac{2}{3})^n \: \: \forall \: \: n \geq 0$ 

but for all $c > 0$, we see that there exists a $k_0$ such that $(\frac{2}{3})^k \leq c \: \: \forall \: \: k \geq k_0$, which contradicts the inequality above

From this contradiction, we see that $2^n \not\in \Omega(3^n)$

Therefore, because the function is not in the lower bound of $\Omega(3^n)$, we have proven that $2^n \in \Theta(3^n)$ is false, in turn proving that $2^n \not\in \Theta(3^n)$ is true

\vspace{0.1in}

\item \textbf{Problem 9.28.}
For recurrence $f(0)=1$; $f(n)=nf(n-1)$, compare $f(n)$ with (a)~$2^n$ (b)~$n^n$.

\vspace{0.1in}

\item \textbf{*Problem 9.37.}
A recursive algorithm has a runtime $T(n)$ that depends only on $n$, the input of size.
$T(1)=1$ and for an input-size $n$, the algorithm solves two problems of size $\lfloor n/2\rfloor$
and does extra work of $n$ to get the output.
(\textbf{(v1.1)}~Note ``ceiling'' was corrected to ``floor'' here.)
\begin{enumerate}[(a)]
\item Argue that $T(n)$ satisfies the recursion $T(n)=2T(\lfloor n/2 \rfloor)+n$.

\vspace*{.1in}

To show that T(n) satisfies the recursion $T(n) = 2T(\lfloor n/2 \rfloor)+n$, we can prove by induction

[Base Case] $2T(\lfloor 1/2 \rfloor) + 1$ = $2T(0) + 1 = 2 + 1 = 3$, satisfies the recursion when T(1) = 1  





\item Prove $T(n)\in\Theta(n\log n)$.\\
$[$Hint: Induction to show $n\log_2 n\le T(n)\le2n\log_2 n$ for $n=2^k$ and monotonicity.$]$
\end{enumerate}

\newpage

\item \textbf{*Problem 9.39.}
Use the rule of thumb for nested sums on the bottom of page 118
(i.e.,~You can quickly determine the growth rate of a nested sum using:
growth rate of nested sum~$=$~number of nestings~$+$~order of the summand)
to obtain the asymptotic growth rate for the following sums, and verify by exact computation.
If the rule does not work, why not?
\begin{multicols}{2}
\begin{enumerate}[(a)]
\item $\displaystyle \sum_{i=1}^n \sum_{j=1}^i j$
\item $\displaystyle \sum_{i=1}^n ( i^2 + \sum_{j=1}^n j )$
\item $\displaystyle \sum_{i=1}^n \sum_{j=1}^n \sum_{k=1}^n (i^2+ijk)$
\item $\displaystyle \sum_{i=1}^n \sum_{j=1}^{i^2} j$
\end{enumerate}
\end{multicols}

\vspace{0.1in}

%\item \textbf{Problem 9.43.}
%
%\vspace{0.1in}

%\item \textbf{Problem 9.69-9.72.}
%
%\vspace{0.1in}

\item \textbf{Problem 11.3.}
Give the degree sequences of $K_{n+1}$, $K_{n,n}$, $L_n$, $C_n$, $S_{n+1}$, and $W_{n+1}$.

\vspace{0.1in}

\item \textbf{*Problems 11.5 and 11.15(b).}
A graph is regular if every vertex has the same degree.
Which of these graphs are regular?
\begin{enumerate}[(a)]
\begin{multicols}{7}
\item $K_6$
\item $K_{4,5}$
\item $K_{5,5}$
\item $L_6$
\item $S_6$
\item $W_4$
\item $W_5$
\end{multicols}
\end{enumerate}
A graph is $r$-regular if every vertex has the same degree~$r$. Show:
\begin{enumerate}[(a)]
\setcounter{enumi}{1}
\item If $r$ is odd and $n$ is odd, there is no $r$-regular graph with $n$ vertices.
\end{enumerate}

\vspace{0.1in}

\item \textbf{Problem 11.10.}
Give graphs with these degree distributions, or explain why you can't.
Verify $2|E|=\displaystyle\sum_{i=1}^n \delta_i$.
\begin{enumerate}[(a)]
\begin{multicols}{4}
\item $[5,3,3,2,1]$
\item $[3,2,1,1,1]$
\item $[3,3,2,1]$
\item $[3,3,3,3,3]$
\item $[3,3,3,3,3,3]$
\item $[3,3,2,2,2]$
\item $[4,4,4,4,4]$
\item $[4,4,3,2,1]$
\item $[4,3,3,2,2]$
\item $[3,3,3,2,2]$
\item $[3,3,3,3,2]$
\item $[5,3,2,2,2]$
\end{multicols}
\end{enumerate}

\vspace{0.1in}

\item \textbf{*Problem 11.11.}
In a graph only the two vertices $u,v$ have odd degree.
Prove there is a path from $u$ to $v$.

\vspace{0.1in}

\item \textbf{Problem 11.13.}
Compute the number of edges in the following graphs:
\begin{enumerate}[(a)]
\begin{multicols}{7}
\item $K_n$
\item $K_{n,\ell}$
\item $W_n$
\end{multicols}
\end{enumerate}

\newpage

\item \textbf{Problem 11.15(a,c-d).}
A graph is $r$-regular if every vertex has the same degree~$r$. Show:
\begin{enumerate}[(a)]
\item If $r$ is even and $n>r$, there is an $r$-regular graph with $n$ vertices. (Tinker!)
\setcounter{enumi}{2}
\item If $r$ is odd and $n>r$ is even, there is an $r$-regular graph with $n$ vertices.
\item An $r$-regular graph with $4k$ vertices must have an even number of edges.
\end{enumerate}

\vspace{0.1in}

\item \textbf{*Problem 11.17.}
A graph $G$ has $n$ vertices.
\begin{enumerate}[(a)]
\item What is the maximum number of edges $G$ can have and not be connected? Prove it.
\item What is the minimum number of edges $G$ can have and be connected? Prove it.
\end{enumerate}

\vspace{0.1in}

\item \textbf{*Problem 11.27.}
Every vertex degree in a graph is at least 2.
Prove that there is at least one cycle.

\end{itemize}

\end{document}
